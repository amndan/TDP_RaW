%As a first attempt on the RoboCup atWork \todo{consistent style of RaW} competition, we decided not to try the Precision Placement and Conveyor Belt test. It was a good idea because during this attempt we had big problems with the localization and navigation, which are basic functions. This limited us to try further and more difficult tests. We gain experience and improve our software so we will try this time more difficult tests.

Currently, the team considers three different strategies concerning the localization problem. The first one is using a self developed particle filter algorithm. The second is to use the amcl package from ROS. And the third strategy is to use the self developed ohm\_tsd\_slam algorithm mentioned in chapter~\ref{sec:slam}. In the future, all strategies will be tested and compared.

\subsection{Particle-Filter (TH-Nuernberg)}
The team works currently on an own particle filter algorithm. Its functionality is close to amcl localization, \cite{pf_fox} and \cite{pr}. 

\subsection{Particle-Filter (ROS-AMCL)}
The navigation stack from ROS-System includes a package called amcl (Adaptive Monte Carlo Localization). It provides a particle filter algorithm for robot localization. 

The compatibility of amcl algorithm and our SLAM approach has already been tested. It is possible to record a map with ohm\_tsd\_slam and afterwards localize and navigate in that map via amcl and navigation stack from~ROS.

\subsection{SLAM for Localization}
%SLAM means Simultaneous Localization and Mapping. This is because while recording a map, the robot needs to know its position in the map - so it must locate itself in the map while building the map. 
It is possible to disable the mapping in the ohm\_tsd\_slam package, using a localize only mode. In this mode, the software loads a previously recorded map, localizes the robot using its sensor data but does not alter the map. The main weakness of this approach compared to a particle filter is a possible wrong localization. A particle filter is able to re detect a lost pose using global localization, the ohm\_tsd\_slam package does not provide this feature.

%If we disable the mapping part from our SLAM algorithm, we are able to load a previously recorded map and locate the robot. The problem with this approach is that if the algorithm fails to process one measurement, the localization is lost in most cases and we have to quit the run. Particle filters are able to recover from such failures.